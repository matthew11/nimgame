\begin{thebibliography}{x}
\addcontentsline{toc}{chapter}{\bibname}

\bibitem{bibref:pickup_stones} Thomas Fisher: {\em Simulating the Pick-up Stones game: A dynamic approach}, Department of Computer Science and Electrical Engineering,
University of Maryland Baltimore Country, {\ttfamily \url{ http://www.users.miamioh.edu/fishert4/docs/fisher-algo.pdf}}

\bibitem{bibref:clear_thinking}Flesch, Rudolf (1951). {\em The Art of Clear Thinking.} New York: Harper and Brothers Publishers. 3. oldal

\bibitem{bibref:bouton_nim} Charles L. Bouton: {\em Nim, A Game with a Complete Mathematical Theory}, Annals of Mathematics (\url{http://www.jstor.org/stable/1967631})

\bibitem{bibref:mi_modern}Russell Stuart és Norvig Peter: {\em ARTIFICIAL INTELLIGENCE. A MODERN APPROACH. 2nd Edition},  Pearson Education, Inc (Magyar nyelvű fordítás: {\em \url{https://mialmanach.mit.bme.hu/aima/index}})

\bibitem{bibref:brilliant_nim}Nyerő stratégia és bizonyítása: {\em \url{ https://brilliant.org/wiki/nim/}}

\bibitem{bibref:nim_variants}Nim játék variánsai: {\em \url{https://mialmanach.mit.bme.hu/erdekessegek/nim_jatek}}

\bibitem{bibref:turing}Turing-teszt: {\em \url{https://hu.wikipedia.org/wiki/Turing-teszt}}


\bibitem{bibref:nimbers}Nimberek: {\em \url{https://en.wikipedia.org/wiki/Nimber}, \\
	\url{https://en.wikipedia.org/wiki/Ordinal_number},\\ \url{https://hu.wikipedia.org/wiki/Rendsz\%C3\%A1m_(halmazelm\%C3\%A9let)}}

\bibitem{bibref:minimax}Minimax: {\em \url{https://hu.wikipedia.org/wiki/Minimax_elv} és \cite{bibref:mi_modern}}

\bibitem{bibref:alfabeta}Alfa-Béta vágás: {\em \url{https://hu.wikipedia.org/wiki/Alfa-b\%C3\%A9ta_v\%C3\%A1g\%C3\%A1s} és \cite{bibref:mi_modern}} 

\end{thebibliography}
