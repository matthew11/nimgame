\Chapter{Fejlesztői dokumentáció}
\label{Chap:dokumen}

Ebben a fejezetben kell a hallgatónak leírnia a saját eredményeit. Például ilyennek tekinthető a hallgató által elkészített program leírása, algoritmus leírása, alkalmazási lehetőségek, eredmények. Lehet benne több alfejezet vagy al-alfejezet is. Ezek számozása és a tartalomjegyzékben való megjelenése rögzített. A fejezet címe, azaz a ,,Fejlesztői dokumentáció'' megváltoztatható az eredmények szerint. Ezen fejezetben felhasználható oldalak mennyisége összefüggésben van az eledzel fejezettel (lásd \aref{Chap:tema}. fejezet bevezetését), ugyanis ezen két fejezetnek minimum 25 maximum 60 oldalnak kell lennie.\\

És most már gépelhetjük a szöveget\ldots

\Section{Programkód}
Például a \LaTeX-es forrása a következő is lehet:
\begin{verbatim}
\begin{tabbing}
akkor,\= \\
      \>barátaim, itt a tollam,\\
írjanak\= \\
       \>maguk\= \\
       \>     \>helyettem!
\end{tabbing}
\end{verbatim}

A fenti nyomtatási képe pedig:
\begin{tabbing}
akkor,\= \\
      \>barátaim, itt a tollam,\\
írjanak\= \\
       \>maguk\= \\
       \>     \>helyettem!
\end{tabbing}
