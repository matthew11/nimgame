\Chapter{Téma elméleti kifejtése}
\label{Chap:tema}

\section{Mesterséges intelligencia}

\section{A Nim játék leírása}
A Nim játék egy két játékos által játszható teljes információjú stratégiai játék. A játék egyetlenegyszer fakadóan számos változata, illetve továbbgondolása is létezik. Néhányat a későbbiekben röviden ismertetni is fogok. \\

A játék körökre bontott, azaz a játékosok felváltva teszik meg lépéseiket. A játék másik lényeges tulajdonsága, hogy teljes információjú játék, azaz a játék kezdetétől fogva mindkét játékos rendelkezésére áll az összes a játékra vonatkozó ismeret. Beleértve a szabályokat, és a teljes játékteret.\\

Nim játék esetében minden kör egy, és csakis egy lépésből áll, amit az éppen soron következő játékosnak kötelezően meg kell tennie. A játéktér tetszőleges számú halomból állhat, melynek elemeinek darabszáma csakugyan kötetlen (lehet egyforma, és akár mindegyik halom eltérő elemszámú). Hagyományosan ezek az elemek kavicsok, de igazából matematikai szempontból ezen entitások manifesztációja lényegtelen. Mindegyik lépés abból áll, hogy az éppen soron következő játékos az egyik nem üres elemszámú halomból elvesz legalább egy, legfeljebb az adott halom elemszámával megegyező darab (tehát az egész halmot) entitást a halomból.\\

A játék célja az, hogy amikor sorra kerülünk, akkor ne legyen már több halom, azaz az ellenfelet olyan helyzetbe hozzuk, hogy az végső lépést õ teszi meg, az utolsó entitás(okat) õ veszi el. Ez egyébként a leggyakrabban játszott Nim változat, Mis?re néven is ismert. Mint már említettem a Nim játéknak számos változata létezik, így előfordul, hogy fordítva játsszák, azaz nem az a soron következő játékos nyer, aki nem tud lépni, hanem az, aki a végső elem(eket) elveszi az utolsó halomból.\\


\section{A Nim játék története}
A Nim játék különböző variációit nagyon régóta játsszák. Pontos információink nincsenek, de egyes források arra engednek következtetni, hogy már az ókori Kínában is játszották ezt a játékot. Ugyancsak erre enged következtetni a \begin{CJK*}{UTF8}{gbsn}捡石子\end{CJK*}
(jiǎn-shízi) kínai eredetű játék, amely kísértetiesen hasonlít a Nim játékra, azzal a kivétellel, hogy ott egy halommal játsszák, igaz ennek is sok variánsa létezik, és az érvényes lépéseknek a szabályai bonyolultabbak. \\
Európában először a 16. század kezdetén tesznek róla említést, de igazán a figyelem középpontjába csak a 19. század végén került, amikor Charles L. Bouton tanulmányozta, majd 1901-ben a játék teljes elméletét kidolgozta. Úgy tudni a játékot is ő keresztelte el Nimnek, a német "Nimm" (elvenni) szó alapján. Más források arra hívják fel a figyelmet, hogy a "NIM" szót 180 fokkal elfordítva az angol "WIN" (nyerni) szót kaphatjuk meg. \\
A játék további ismertségre tett szert az 1939-es New York -i világkiállításon, ahol az 1886-ban alapított amerikai Westinghouse Electric Corporation cég bemutatta a Nimatront, amely Nim játékot játszott. A dolog külön érdekessége, hogy ez volt az első elektromos számítógépes játék.

\section{Ismertebb Nim variációk}


\section{A Nim játék matematikai háttere}

\section{Nyerő stratégia}

\section{Northcott-sakk}
