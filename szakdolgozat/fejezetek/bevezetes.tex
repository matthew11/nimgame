%Az elsõ fejezet
\Chapter{Bevezetés}
\label{Chap:bevezetes}


Az emberiséget már jó ideje foglalkoztatja, hogy megértse saját gondolkodásának a működését Hogy megismerje hogyan gondolkodik, miként rendszerezi, és használja fel a megszerzett tudást. Bár a mesterséges intelligencia - mint különálló tudományág - viszonylag fiatal, mégis az utóbbi idők technológiai fejlődése tette igazán lehetővé ennek a tudománynak a gyakorlati felhasználását. \ujsor

A számítógép megjelenése volt az, ami életre hívta ezt a tudományágat, hiszen lehetővé tette az embernek, hogy önmagától elvonatkoztatva, egy különálló entitáson vizsgálja a gondolkodás tudományát. Az utóbbi idők robbanás-szerű fejlődése - olyan tudományterületeken, mint például biológia, elektronika, matematika - nem csak intelligens programok megírását tette lehetővé, hanem a gépek tároló, és feldolgozó kapacitásának ugrás-szerű növekedése elérhető közelségbe hozta az egyre valósághűbb, intelligens ágensek elkészítését, és azok futtatását. \ujsor

A Mesterséges Intelligencia egy olyan tudomány ág, amivel nem csak érdemes foglalkozni, hanem szükségszerű is. Legújabb korunkat megfigyelve észrevehető az a tendencia, hogy az idő múlásával egyre több helyen, és egyre nagyobb mértékben hagyatkozunk a gépek segítségére, a gépek által elvégzett munkára. Jelenleg semmi sem utal arra, hogy ez a jelenség megváltozik a jövőben, ha pedig továbbra is ebbe az irányba haladnak a dolgok, akkor egy idő után az ember nem lesz képes a számítógépes rendszerek vezérlésére, mikromenedzselésére, és ezen feladatokat is rá kell bíznia a gépekre, mégpedig egy intelligens szoftverre. \ujsor

Erre a jelenségre már ma is sok példa áll rendelkezésre. Vegyük például a ma egyre inkább népszerű elképzelést, az IoT-t, vagyis a dolgok internetjét. Ezen elképzelés szerint a (közeli) jövőben a hétköznapi használati tárgyaink jelentős része (óra, mérleg, telefon, ruhaszekrény, gépjárművek, lámpák) átalakul "okos" eszközzé, amelyek egymással kapcsolatban állnak, egyszóval kommunikálnak. Ekkora mennyiségű kommunikációra viszont nagyon is nehéz felkészíteni a telekommunikációs infrastruktúrát, illetve sok esetben nem is lehet. Gondoljunk csak abba bele, hogy az emberek (okos eszközeikkel együtt) folyamatos mozogásban vannak. Ingáznak munkába, rendezvényekre mennek, utaznak, egyszóval élnek. Ez az infrastruktúrát nem egyenletesen terheli, sokszor bizonyos részeire hirtelen nagy mértékű terhelést ad, amire egy ember képtelen megfelelő gyorsasággal, és hatékonysággal reagálni. Erre a problémára fog megoldást nyújtani a SDN (Software Defined Network), amely egy olyan komplex hálózati megoldás, ami folyamatosan monitorozza a telekommunikációs infrastruktúrát, az esetlegesen bekövetkező váratlan eseményre reagál, és megfelelően helyes döntést hozva a rendszer túlterhelt részeit további erőforrásokkal megtámogatva tehermentesíti. \ujsor

A mesterséges intelligenciáknak gyakorlati felhasználásának csak a képzelet szab határt, és talán - gyakorlatias ember lévén - éppen ezért foglalkoztat engem is a téma. Régebben már ugyan készítettem kétszemélyes játékot, amibe gépi játékost is terveztem, de akkor még nem jutottam el odáig, hogy ezt meg is valósítsam. Most egy igen egyszerű játékkal, a Nim-mel ezen régi adósságomat - önmagammal szemben is - törlesztem.




